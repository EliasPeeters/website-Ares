\section{Einleitung}
Über vier Jahre hinweg verlagerten sich Teile des politischen Geschehens der USA auf Social-Media-Plattformen. Der Präsident der USA informierte die Bürger durchschnittlich mehrmals täglich \autocite[vgl.][]{TrumpAllTweets} auf seinen privaten Social-Media-Accounts darüber, inwiefern er als Präsident der USA das politische Geschehen in seinem Land veränderte. Social Media, hauptsächlich in Form von Twitter, stellte also das Sprachrohr eines der mächtigsten Männer der Welt \autocite[vgl.][]{MächtigTrump} dar. Knapp vier Jahre nach Amtsantritt entzogen ihm selbige Plattformen noch während seiner Präsidentschaft die Möglichkeit, dort seine Meinung kundzutun \autocite[vgl.][]{BanTrumpFBInsta}\autocite[vgl.][]{BanTrumpTwitter}. Damit lösten sie weltweite Diskussionen aus, die sich mit dem politischen Einfluss der Konzerne, die die Plattformen betreiben, befassen.
In dieser Hausarbeit wird zuerst erläutert, inwiefern Social Media die politische Meinungsbildung beeinflussen kann. Zu Beginn wird die politische Nutzung von Social Media vor allem in Deutschland analysiert. Es folgt eine Betrachtung der Beeinflussungsmöglichkeiten auf Social Media, die durch das Beispiel der US-Präsidentschaftswahl 2016 verdeutlicht werden. Dem schließt sich eine Prognose der Beeinflussung über Social Media für die deutsche Bundestagswahl 2021 an. Die im Laufe der Untersuchungen aufgezeigte Beeinflussung der Politik mündet in die Leitfrage der Hausarbeit: Sind die Social-Media-Plattformen sowohl rechtlich als auch ethisch dafür verantwortlich, dass die politische Meinungsbildung durch ihren Einfluss verändert wird?

\section{Begriffsdefinitionen und Methodik}
\subsection{Politische Meinungsbildung}
Zur Beantwortung der Frage müssen zuerst einige Begriffe definiert werden. Hierbei ist unter anderem der Begriff der politischen Meinungsbildung zu betrachten.
In der Demokratie (gr. dēmos = Volk), also einer Volksherrschaft \autocite[vgl.][]{BegriffDemokratieBPB}, gilt das Prinzip der Verleihung von Mandaten auf Zeit. Das Wahlvolk bildet sich eine politische Meinung, gibt seine Stimme für eine seiner Meinung nach geeigneten Volksvertretung ab und lässt sich durch diese vertreten \autocite[vgl.][]{VolksvertretungBPB}. Somit gilt  "Our Government rests in public opinion" \autocite[]{SpeechALincoln}, wie Abraham Lincoln einst sagte. Das Grundgerüst der Demokratie sind also legitime Eigeninteressen der Wähler, die sich zum Teil überschneiden und in Folge für das Zusammenschließen zu Interessengruppen, also beispielsweise Parteien, sorgen. Die Bildung einer politischen Meinung ist dabei dynamisch. Durch den Austausch in der politischen Öffentlichkeit setzen sich alle Akteure mit den Standpunkten anderer auseinander und können diese Standpunkte übernehmen, diskutieren, ablehnen oder zu einem Kompromiss zusammenführen. Um den effizienten Austausch in einer pluralistischen Demokratie mit einer hohen Anzahl an Teilnehmern zu ermöglichen (in Deutschland waren 2017 61,69 Millionen Menschen wahlberechtigt \autocite[vgl.][]{WahlberechtigteBPB}), existieren Massenmedien, die Ideen für Beschlüsse und andere Informationen mit ihrer großen Reichweite vermitteln.
Diese Medien haben sich im Laufe der Zeit gewandelt und es sind auch neue Medien hinzugekommen. Die jüngste Ergänzung zu Presse, Hörfunk und Fernsehen \autocite[vgl.][]{MassenmedienBPB} ist das Internet und infolgedessen die "sozialen Medien", welche im Anschluss definiert werden.

\subsection{Social Media}
Im Folgendem wird der Begriff \glqq Social Media\grqq{} als diejenigen digitalen Plattformen verstanden, auf denen Menschen oder Organisationen Netzwerke erstellen und über diese auf elektronischem Weg Informationen austauschen können \autocite[vgl.][S. 1]{DefSM}\autocite[S. 62-64]{kaplan2010}.
Der Fokus in dieser Hausarbeit liegt auf den sozialen Medien, die weltweit am häufigsten für Nachrichten verwendet werden \autocite[vgl.][S.~30]{DNR2020}. Hierbei wird der Messengerdienst WhatsApp ausgeschlossen, da dieser seit einer Änderung der Nutzungsbedingungen 2019 \autocite[vgl.][]{WANutz} als Primärquelle für Nachrichten zu vernachlässigen ist. Dementsprechend werden ausschließlich Facebook, YouTube, Twitter und Instagram in der weiteren Analyse betrachtet. 
Social Media erlangte in den letzten Jahren eine zunehmende Bedeutung, was sich vor allem an den Nutzerzahlen weltweit erkennen lässt, die sich von 2,1 Milliarden in 2015 auf 4,2 Milliarden in 2021 verdoppelt haben \autocite[vgl.][]{UsersWorldwideSM}. 

\subsection{Verantwortung}
Des Weiteren muss für die Beantwortung der Leitfrage der Begriff Verantwortung definiert werden. Der Duden beschreibt Verantwortung als die \glqq Verpflichtung, dafür zu sorgen, dass (innerhalb eines bestimmten Rahmens) alles einen möglichst guten Verlauf nimmt, das jeweils Notwendige und Richtige getan wird und möglichst kein Schaden entsteht\grqq{} \autocite[vgl.][]{DudenVerantw}.  Im Fall der Verantwortung der Social-Media-Konzerne muss also betrachtet werden, was sowohl ethisch als auch rechtlich gesehen \glqq das Richtige\grqq{} ist. Hierfür sind jeweils auch die Konsequenzen der Handlungen zu betrachten. So muss für die Handlungsmöglichkeiten und -vorschriften jeweils erörtert werden, ob und inwiefern hierbei ein Schaden für den Konzern, die betroffenen Personen oder die Gemeinschaft entsteht (s. Kapitel 4).

\subsection{Einführung in die Umfrage}
Um erste Eindrücke zur Nutzung von Social Media als Quelle politischer Informationen und Nachrichten zu erlangen, wurde von uns eine anonymisierte Umfrage durchgeführt. Die Fragen werden im Weiteren in den Kapiteln erläutert, in denen die Ergebnisse für die Analyse genutzt werden. Vorab sei gesagt, dass von den vier Fragen nur die zwei für die Untersuchungen genutzt werden, die für die Betrachtungen am aussagekräftigsten und relevantesten waren. Die Umfrage wurde mit Google Forms erstellt und lieferte zum heutigen Zeitpunkt nach einer Laufzeit von ungefähr einer Woche 416 Antworten. Der vollständige Umfragebogen liegt im Anhang bei (\autocite[]{fig:umfrage}). Die Umfrage wurde über digitale Kanäle wie E-Mail, WhatsApp, Instagram, Microsoft Teams und Twitter verbreitet. Die Repräsentativität ist durch diese Art der Verbreitung bereits leicht eingeschränkt, da nur Personen zum Themenfeld \glqq Politik und Social Media\grqq{} befragt wurden, die auch Social Media nutzen. Personen, die kein Social Media nutzen, hätten unter Umständen anders geantwortet. Außerdem reduziert sich das Teilnehmerfeld auf Studenten, Mitarbeiter der Praxisunternehmen, ehemalige Mitschüler und Freunde sowie deren Familien. Dies bedeutet, dass nicht das gesamte Bevölkerungsspektrum abgedeckt wird, da die Teilnehmergruppe meist aus höheren Bildungsschichten und überwiegend aus dem lokalen Umfeld stammt. Zudem gibt es starke Unterschiede bei der Teilnehmeranzahl in den verschiedenen Altersgruppen. Rund 75\% der Befragten sind zwischen 10 und 30 Jahren alt. Somit fehlen in den höheren Altersschichten Daten, wodurch eine empirische Betrachtung über alle Altersgruppen hinaus nur bedingt ermöglicht wird. Nichtsdestotrotz liegt eine ausgewogene Verteilung der Geschlechter vor und die hohe Teilnehmeranzahl der jüngeren Altersgruppen verleiht den Daten in diesem Altersspektrum eine repräsentative Gültigkeit (\autocite[]{fig:geschlecht}). Für die anderen Generationen lassen sich zwar keine absoluten Aussagen treffen, in Kombination aller Antworten ergeben sich allerdings Trendlinien, die in das Gesamtgefüge passen.

Bei Fragen, bei denen der Teilnehmer seine Einschätzung auf einer Skala treffen sollte, wurde diese stets von null bis fünf gewählt, sodass durch einen nicht vorhandenen Mittelwert mindestens eine tendenzielle Positionierung stattfinden musste.

\section{Nutzung von Social Media zu politischen Zwecken}

\subsection{Entwicklung des politischen Geschehens auf Social Media}
Um die Kernfrage in Bezug auf die politische Verantwortung der Social-Media-Konzerne beantworten zu können, ist es unerlässlich, ein politisches Geschehen auf Social Media festzustellen und dessen Relevanz zu analysieren. Hierbei stellt sich zuerst die Frage, aus welchem Grund politische Akteure im Internet und auf Social Media auftreten sollten.

Da die politische Meinungsbildung und das System der Demokratie auf dem Austausch von Meinungen beruhen (s. Kapitel 2.1), muss jedem Bürger freier Zugriff auf die Informationen des politischen Geschehens ermöglicht werden. Hierfür werden Massenmedien benötigt, deren Nutzung sich in den letzten 50 Jahren gewandelt hat. Dies zeigt eine Zusammenfassung der Langzeitstudie von ARD und ZDF zur Mediennutzung in der BRD \autocite[vgl.][]{Mediennutzung2015}. Während zwischen 1970 und 2005 die Nutzung von Radio und Fernsehen kontinuierlich angestiegen ist, sind die Werte bei diesen beiden Medien ab 2005 bis 2021 um durchschnittlich 14\% gesunken. Gleichzeitig ist die Nutzung von Tageszeitungen über den gesamten Zeitraum hinweg um 34\% gesunken. Die stärkste Entwicklung ist allerdings beim Internet festzustellen, welches
erstmals 1995 mit einer Nutzungsdauer von unter zehn Minuten in der Studie erwähnt wurde und 2015 eine durchschnittliche tägliche Nutzungsdauer von 107 Minuten hatte. Es ist damit 2015 das Medium, das nach Fernsehen und Radio am häufigsten genutzt wird. Diese Steigerung hat sich seitdem fortgesetzt, wie der aktuelle Bericht der Langzeitstudie zeigt \autocite[vgl.][]{Mediennutzung2020}. Demnach ist der Prozentsatz der Bürger, die das mediale Internet nutzen, erneut von 44\% (2019) auf 50\% (2020) angestiegen \autocite[vgl.][2]{Mediennutzung2020PDF}. 
In einer Umfrage der Bundeszentrale für politische Bildung wurden die Motive für die Mediennutzung erfragt \autocite[vgl.][]{NutzungMedienBPB}. Demnach ist der Anteil derer, die Zeitungen nutzen, um sich zu informieren, von 32\% (2010) auf 26\% (2015) gesunken. Im gleichen Zeitraum ist dieser Wert für das gesamte Internet, also nicht nur Social Media, von 29\% auf 34\% gestiegen. Ähnliches zeigt der Digital News Report von 2020, der sich mit den \glqq Sources of News\grqq{}  \autocite[vgl.][71]{DNR2020} befasst. Laut diesem ist Social Media als Teil des Internets als Informationsquelle von 18\% (2013) auf 37\% (2020) gestiegen und hat somit einen höheren Wert als die Printmedien erreicht, die von 63\% (2013) auf 33\% (2020) gesunken sind.

Aus den steigenden Nutzungszahlen lässt sich eine zunehmende Relevanz des Internets feststellen, die bereits einen Grund für das Auftreten von Politikern und Parteien im Internet und auf Social Media darstellt. Hierdurch können sie ihre Reichweite vergrößern und Bevölkerungsgruppen erreichen, die andere Massenmedien nicht oder weniger nutzen. Als Beispiel für letzteres ist die jüngere Bevölkerung (wir betrachten hierfür die Gruppe der 14- bis 24-Jährigen) anzuführen. 
In der Umfrage (s. Kapitel 2.4) wurde erfragt, inwiefern die Teilnehmer politische Nachrichten über Social Media auf einer Skala von 0 (kein Konsum auf Social Media) bis 5 (ausschließlich Konsum über Social Media) konsumieren. Auch wenn die Teilnehmeranzahl mit höherem Alter absinkt, lässt sich der Trend erkennen, dass der Anteil des Konsums der Nachrichten auf Social Media bei den jüngsten Teilnehmern am höchsten ist und für ältere Gruppen kontinuierlich sinkt. Für die Altersgruppe zwischen 14 und 24 Jahren (251 Teilnehmer) wurde ein Durchschnittswert von 2,66 ermittelt. Demnach konsumiert also diese Gruppe mehr als die Hälfte (53\%) ihrer politischen Nachrichten über Social Media. Somit liegt das Interesse von Parteien und Politikern an einer Präsenz auf Social Media darin begründet, dass sie dort auch die Jugendlichen und jungen Erwachsenen, die die zukünftige Wählerschaft ausmachen, erreichen.

Unter anderem aus diesen Gründen eröffneten die Parteien, die bei der letzten Bundestagswahl mehr als 1\% der Stimmen erlangten, überwiegend in den Jahren 2008/09 oder spätestens 2013 Accounts auf Twitter und Facebook und erreichten bis zu 550.000 Follower auf einer Plattform (\autocite[]{tab:parteienSocialMedia}). Die deutsche Politik ist also auch auf Social Media vertreten und baut den Einfluss dort weiterhin aus, was sich anhand der steigenden Followerzahlen auf Twitter zeigen lässt (\autocite[]{tab:parteienSocialMedia}). 

Nachdem die steigende Relevanz von Social Media unter anderem für politische Zwecke in Deutschland gezeigt wurde, soll die Analyse auf ein anderes Land ausgeweitet werden. Hierfür eignen sich die USA, da diese zum einen die älteste Demokratie der Welt \autocite[vgl.][]{ÄltesteDemokrUSA} sind und zusätzlich aufgrund der Social-Media-Präsenz des Ex-Präsidenten Donald Trump des Öfteren in der Öffentlichkeit standen. So ergibt beispielsweise eine gemeinsame Suche der Wörter \glqq Trump\grqq{} und \glqq Twitter\grqq{} alleine auf tagesschau.de 1382 Suchresultate (am 10.02.2021) \autocite[vgl.][]{SearchTagesschau}. 
Die Steigerung der Präsenz Donald Trumps auf Social Media zeigt sich auch im Vergleich der Followerzahlen zur US-Wahl 2016 mit denen zu Beginn 2021. Während der Ex-Präsident 2016 noch 11,9 Millionen Follower hatte \autocite[vgl.][]{2016TrumpTwitter}, waren es am 10.01.2021 vor seiner Sperrung 88,8 Millionen \autocite[vgl.][]{2021TrumpTwitter}. Auch der Twitter-Kanal des jetzigen US-Präsidenten Joe Biden hatte Anfang Februar 2021 bereits 27,9 Millionen Follower \autocite[vgl.][]{JoeBidenTwitter}. Vergleicht man diese Zahlen mit den Followerzahlen deutscher Parteien pro 100.000 Einwohner (\autocite[]{tab:trumpdeutschland}), wird deutlich, dass die US-Politiker bezüglich der Followerzahlen auf Social Media stärker vertreten sind. 

Eine ähnlich hohe Relevanz von Social Media für Politik in den USA ist in einer Umfrage in den USA aus dem Herbst 2019 zu erkennen: Dort stellen die sozialen Medien (18\%) nach Fernsehen (45\%) und News-Websites oder News-Apps (25\%) die Quelle dar, aus der die Teilnehmer hauptsächlich ihre politischen Nachrichten konsumieren \autocite[vgl.][]{AmericansSourcesNews}.  Bei diesem Ergebnis sei gesagt, dass die Teilnehmer nur eine Antwort auswählen konnten, während bei ähnlichen Umfragen in Deutschland (s.o) mehrere Antworten möglich waren. Social Media ist in den USA also auch von hoher Relevanz für die politische Meinungsbildung.


\subsection{Social Media und politische Meinungsbildung - Einflussmöglichkeiten}
Nach der Feststellung, dass die sozialen Medien für die politische Öffentlichkeit und damit für die politische Meinungsbildung zunehmend an Relevanz gewonnen haben (s. Kapitel 3.1), ist es nun unerlässlich zu betrachten, was die Meinungsbildung auf Social Media auszeichnet und gegebenenfalls von der Meinungsbildung durch die etablierten Massenmedien (s. Kapitel 2.1) unterscheidet. Dabei gilt besonders zu untersuchen, inwiefern Einfluss auf den Prozess der Meinungsbildung über Social Media genommen wird. Hierfür ist zuerst das Definieren des Begriffes \glqq Einfluss\grqq{} notwendig. Die Definition orientiert sich dabei an denen vom Cambridge Dictionary und vom Duden \autocite[vgl.][]{einflussduden}\autocite[vgl.][]{influencecambridge}. Im Folgenden wird unter dem Begriff \glqq Einfluss\grqq{} eine Wirkung auf eine Zielperson oder Personengruppe verstanden, die das Potenzial hat, das Verhalten oder die Meinung der Zielperson oder Personengruppe zu verändern. Zu beachten ist, dass die Einflussnahme passiv (unbeabsichtigt) oder aktiv (beabsichtigt) erfolgen kann. Weiterhin wird der Begriff \glqq Beeinflussung\grqq{} synonym zum \glqq Einfluss\grqq{} verwendet.   

Generell gilt, dass alle Arten von Medien einen Einfluss auf die politische Meinungsbildung eines Adressaten haben, da sie die Bandbreite an Informationen, die vermittelt werden, selektieren. So müssen sie aus wirtschaftlichen Gründen (zum Beispiel begrenzter Platz in Zeitungen) nach eigenem Ermessen entscheiden, welche Nachrichten von größter Bedeutung sind (sogenanntes \glqq Agenda-Setting\grqq{} \autocite[vgl.][]{agendasetting}). Dieses Agenda-Setting kann also bereits die Informationen, die für die politische Meinungsbildung zur Verfügung stehen, filtern und damit die letztendliche Meinung beeinflussen. 

Was Social Media unter anderem von den anderen etablierten Massenmedien unterscheidet, ist der Unterschied in der Informationsselektion. Bei den etablierten Medien entscheidet der Nutzer zwar, welche Zeitung er kauft oder welche Sendung er schaut, der Zeitungsstand, an dem er kauft, bietet jedoch allen Kunden eine ähnliche Auswahl an. Anders ist es auf Social Media, wo die angebotenen Inhalte auf den Nutzer und dessen Interessen zugeschnitten sind. Es handelt sich metaphorisch gesprochen um einen personalisierten Zeitungsstand, der für den Nutzer kuratiert wird. Ermöglicht wird dies durch Algorithmen von den jeweiligen Social Media Plattformen, welche Beiträge auswählen, die für den Nutzer von möglichst hoher Relevanz sind \autocite[vgl][315]{socialmediamining}. Die Streaming-Plattform YouTube ordnet ähnliche Videos hierbei durch Co-Visitations, also das Nacheinanderschauen von Videos, zusammen und schlägt diese Nutzern mit ähnlichen  Interessen vor.

Das Grundprinzip dieser Algorithmen ist das Social Media Mining, bei dem es darum geht, möglichst viele Informationen über den Nutzer zu sammeln und auszuwerten \autocite[16]{socialmediamining}. Die Vorgehensweise ist dabei in die drei Schritte des Darstellens, Analysierens und schlussendlich des Herausarbeitens sogenannter \glqq patterns\grqq{} \autocite[vgl.][21]{socialmediamining}, welche wiederkehrende Muster beim Nutzerverhalten sind, gegliedert. Als Datengrundlage dienen hierbei sowohl soziale Interaktionen des Nutzers \autocite[vgl.][21]{socialmediamining}, als auch grundsätzliches Verhalten im Internet. Zum Beispiel kann aus dem Rhythmus der Tastenanschläge mit Hilfe technischer Analysen die Konsumneigung des Nutzers ermittelt werden. \autocite[vgl.][S.2]{martini2017algorithmen}. Das Ziel des Social Media Minings ist es, die Nutzer aufgrund ihrer Eigenschaften in Netzwerke zu sortieren \autocite[vgl.][29]{socialmediamining}. In diesem Netzwerk kann zum Beispiel dargestellt werden, ob sich zwei Nutzer kennen oder nicht (\autocite[]{fig:graphNetwork}). Zusätzliche Gewichtungen an den Verbindung können zudem die Wahrscheinlichkeiten der Bekanntschaft darstellen, wodurch detailliertere Aussagen getroffen werden können \autocite[vgl.][66]{socialmediamining}. In den Netzwerken können generell Menschen mit ähnlichen zugeordneten Eigenschaften, wie der Konsumneigung, dargestellt werden. Dabei schließen die Algorithmen auf Basis von Gruppenwahrscheinlichkeiten konkrete Aussagen über einzelne Nutzer \autocite[vgl.][S.4]{martini2017algorithmen}. Diese Analysen ermitteln somit Korrelationen, jedoch keine Kausalitäten \autocite[vgl.][9]{martini2014big}. 

Da die Basis dieser Netzwerke Eigenschaften sind, können hiermit dem Nutzer gezielt personalisierte Inhalte vorgeschlagen werden. Die bereits genannten Netzwerke, in denen die Bekanntschaft zwischen Nutzern dargestellt wird, können somit für neue Freundesvorschläge verwendet werden \autocite[vgl.][314]{socialmediamining}. Mithilfe dieser Netzwerke und den Informationen über den Nutzer kann nun auch das Microtargeting stattfinden, bei dem es darum geht, Menschen gezielt Wahlwerbung zu zeigen. Diese Methode wurde in der US-Präsidentschaftswahl 2016 eingesetzt und wird im folgenden Kapitel genauer erläutert.

Die Netzwerke haben jedoch auch ohne Werbung einen Einfluss auf das Verhalten des Nutzers. Da die Nutzer in dem Netzwerk einer Gruppe von sehr ähnlichen Menschen ausgesetzt sind und Menschen dazu neigen, Eigenschaften von anderen zu adaptieren, werden sich die Nutzer in diesen Netzwerken immer ähnlicher \autocite[vgl.][285]{socialmediamining}. Hinzu kommt die Theorie der Schweigespirale, die besagt, dass Menschen weniger gewillt sind, ihre Meinung zu äußern, wenn sie das Gefühl haben, in der Minderheit zu sein \autocite[vgl.][1]{hampton2014social}. Daraus folgt, dass Nutzer eine Gegenmeinung in einem Netzwerk nicht äußern und der Gegenschluss ist somit auch, dass Nutzer erst recht ihre Meinung äußern, wenn sie sich damit in der Mehrheit fühlen. Aus diesen Gründen kann in einem Netzwerk eine verzerrte Wahrnehmung von Meinungen entstehen. 

Außerdem muss gesagt werden, dass vollständige Objektivität in den Medien einen Idealzustand darstellt, der nur zu einem bestimmten Maße erreicht werden kann \autocite[vgl.][]{journalismusobjektivitaet}. Auch hierdurch kann eine Beeinflussung des Adressaten stattfinden. Auf Social Media Plattformen kommt dazu, dass zudem auch die Algorithmen nicht vollständig objektiv und neutral die Netzwerke bilden können, da sie von einem Unternehmen entwickelt werden, welches den ökonomischen Wert höher stellt als die Wertvorstellung der Bevölkerung \autocite[vgl.][3]{martini2017algorithmen}.

Ein weiterer Unterschied zwischen Social Media und den etablierten Medien ist die Interaktion des Nutzers. Bei Social Media können diese direkt und in Echtzeit ihre Meinung zu Beiträgen Anderer durch Likes oder Kommentare kundtun. Daher werden Plattformen wie Twitter auch als \glqq microblogging social media\grqq{} bezeichnet. Ein hierbei auftretendes Problem ist die Präsenz von Bots, also automatischen digitalen Werkzeugen, in sozialen Netzwerken. Diese können gezielt verwendet werden, um zum Beispiel soziale Netzwerke mit ähnlichen Nachrichten zu fluten, um damit mithilfe der Theorie der Schweigespirale Gegenmeinungen möglichst zu unterbinden. Dieses Phänomen wurde während der US-Wahl 2016 verwendet und wird im folgenden Kapitel genauer erläutert.


\subsection{Fallbeispiel US-Wahl 2016}
Nachdem die verschiedenen Möglichkeiten der Einflussnahme auf die politische Öffentlichkeit erläutert wurden, ist es an der Zeit, diese an praktischen Beispielen zu verdeutlichen. Hierfür werden die aussagekräftigsten Ereignisse rund um die US-Präsidentschaftswahl aus dem Jahr 2016 betrachtet. Viele Geschehnisse aus dem Jahr sind wegweisend und haben aufgezeigt, welche Risiken die Verknüpfung von Social Media und Politik birgt (nicht zuletzt ist der Begriff „Fake News“ im Jahr 2016 populär geworden) \autocite[vgl.][]{BBCHistoryFN}\autocite[vgl.][]{WashHistoryFN}\autocite[vgl.][]{NatoFN}.

Eine der Hauptquellen in diesem Kapitel wird das Buch „Mindf*ck“ (2019) von Christopher Wylie sein \autocite[vgl.][]{wylie2019mindf}, einem ehemaligen Angestellten der Firma Cambridge Analytica, über die er als Whistleblower nach seinem Ausstieg vor mehreren politischen Organen in verschiedenen Staaten aussagte (wie zum Beispiel dem Kongress der USA \autocite[vgl.][]{CongressWylie}). Es sei gesagt, dass es sich bei dem Werk um keine wissenschaftliche Arbeit handelt und der Inhalt nicht peer-reviewed ist. Das Buch wurde aus einer Perspektive geschrieben, die der republikanischen Wahlkampagne 2016 gegenüber skeptisch, wenn nicht sogar negativ eingestellt ist. Daher ist der Ton nicht durchgehend sachlich und durch Wylies persönlichen politischen Standpunkt verfälscht. Da die meisten Daten, die in dem Buch genannt werden, sich jedoch mit denen anderer Quellen decken, ist es zulässig, Sachinformationen aus dem Werk zu entnehmen, wenn diese vorsichtig auf Richtigkeit geprüft werden (unter anderem durch den Abgleich mit anderen Quellen).

Der erste Präzedenzfall, anhand dessen die Einflussmöglichkeiten auf die politische Meinungsbildung verdeutlicht werden, bezieht sich auf die Arbeit von der Firma Cambridge Analytica (im Folgenden mit „CA“ abgekürzt). CA wurde engagiert, um republikanische Kandidaten zu unterstützen. Dabei konzentrierte CA sich auf politisches Microtargeting, das heißt das Schalten politischer Anzeigen, die auf den Nutzer zugeschnitten sind \autocite[vgl.][]{MicrotargetingBPB}. Zugeschnitten werden die Anzeigen mithilfe einer ausführlichen Datenbasis über die Wähler. Diese Datenbasis erlangte CA zum Teil durch das Sammeln von Facebook-Nutzerdaten wie demographische Daten, Aktivitäten und Likes, aber auch aus staatlichen und kommerziellen Quellen \autocite[vgl.][]{CAdataInsider}.

Staatliche Quellen sind zum Beispiel Verzeichnisse über Bürger mit Jagd-, Angel- oder Waffenlizenzen, kommerzielle Quellen sind Datenhändler mit Auskünften über Informationen wie Finanzstatus oder Flugmeilen eines Bürgers \autocite[vgl.][120,121,122]{wylie2019mindf}.  
Schlussendlich verfügte CA über die Facebook-Daten von bis zu 87 Millionen Nutzern \autocite[vgl.][34]{wylie2019mindf}\autocite[vgl.][]{BBC87Million} und erstellte in Kombination mit den anderen Datenquellen Persönlichkeitsprofile. Das Ergebnis: Man konnte Korrelationen zwischen dem Persönlichkeitsprofil und dem Wahlverhalten feststellen und somit für jedes Profil die Wahrscheinlichkeit bestimmen, mit der eine Person für einen bestimmten Kandidaten abstimmen würde \autocite[vgl.][42]{wylie2019mindf}. Als anschauliches Beispiel: Besitzt ein Wähler eine Waffe, so ist es wahrscheinlicher, dass er für einen republikanischen Kandidaten abstimmen wird \autocite[vgl.][]{StatistaGun}. Zur Verdeutlichung der Genauigkeit dieser psychologischen Profile: Nach einer Studie können Computermodelle anhand von Facebook-Likes psychologische Profile erstellen, die Menschen besser einschätzen als andere Menschen. Bei 10 Likes wird die Persönlichkeit vom Modell besser vorhergesagt als von Kollegen, bei 150 Likes besser als von einem Familienmitglied und bei 300 Likes besser als vom eigenen Ehepartner \autocite[vgl.][]{PersonalityJudgements}. 

Mithilfe dieser sehr genauen Persönlichkeitsprofile, ist es also möglich gewesen, die Wähler sehr gezielt anzusprechen und effektiver zu beeinflussen, da Wähler generell stärker von Anzeigen überzeugt werden, die ihren eigenen Persönlichkeitsmerkmalen entsprechen \autocite[vgl.][]{zarouali2020using}. 
Es war CA aber nicht nur möglich, damit gezielt Trump-Wähler zu mobilisieren und die „Republicans“ zu stärken, sondern zusätzlich wurde versucht, den politischen Gegner, die „Democrats“, zu schwächen. So wurden Wählergruppen, die tendenziell eher Hillary Clinton gewählt hätten, wie Frauen oder Afroamerikaner, mit negativen Informationen zu Hillary Clinton zum Zweifeln gebracht, ob sie überhaupt wählen gehen sollten \autocite[vgl.][]{MicrotargetingBPB}\autocite[vgl.][232,233]{wylie2019mindf}. Ferner wurden Indizien dafür gefunden, dass vor allem Afroamerikaner gezielt in den sogenannten „Swing States“ demobilisiert wurden \autocite[vgl.][]{2016AfroVotesGuardian}. Somit ist die auf Afroamerikaner gerichtete Desinformationskampagne gegen Hillary Clinton wahrscheinlich einer der Gründe dafür, warum die Wahlbeteiligung der Afroamerikaner das erste Mal seit 20 Jahren gesunken war \autocite[vgl.][]{2016AfroVotesGuardian}\autocite[vgl.][]{2016AfroVotesPewResearch}. 

Ein weiterer entscheidender Aspekt der Einflussnahme in der US-Wahl 2016 ist die weite Verbreitung von Fake News, unverifizierten Inhalten und polarisierender oder reißerischer Artikel \autocite[vgl.][]{howard2018social}. Aufschluss darüber geben die Verlinkungen zu externen Websites in Twitter-Posts.  So ergab eine Analyse dazu, was auf Twitter im Zeitraum von acht Tagen vor und drei Tagen nach der Wahl verlinkt wurde, dass 20\% der Links auf Websites mit polarisierenden Artikeln, Verschwörungstheorien, unverifiziertem Wikileaks-Inhalt, ungeprüften russischen Quellen und Desinformationen führten \autocite[vgl.][3]{howard2018social}. Hervorzuheben ist, dass der „seriöse“ politische und journalistische Inhalt (Expertenmeinungen, renomierte Nachrichtensender) lediglich 30\% der Verlinkungen ausmachte, die Links zu „unseriösem“ Inhalt mit einer Quote von 20\% also zwei Drittel der Popularität von seriösem Inhalt erreichten \autocite[vgl.][3]{howard2018social}. Bei den geraden genannten Zahlen handelt es sich um Durchschnittsdaten auf nationaler Ebene, in Michigan beispielsweise war der Anteil der seriösen Inhalte den gesamten Zeitraum über niedriger als der Anteil der unseriösen Quellen \autocite[vgl.][3]{howard2018social}. In einigen Bundesstaaten waren Twitter-Nutzer somit mehr mit desinformierenden Inhalten als mit professionellen journalistischen Nachrichten in Kontakt gekommen. Interessant hierbei ist, dass die Quote von polarisierenden Nachrichten in den Swing States durchschnittlich höher war als in nicht umkämpften Staaten, was auf ein gewisses Targeting seitens der ursprünglichen Verbreiter hinweisen könnte.

Insgesamt lässt sich außerdem zu \glqq Fake News\grqq{} sagen, dass ihre politischen Statements in Summe stark zu Donald Trumps Gunsten ausgefallen sind. Insgesamt wurden 115 Pro-Trump-Fake-News 30 Millionen Mal auf Twitter geteilt, wobei pro-Trump dasselbe bedeutet wie anti-Clinton. Hillary Clinton ist mit nur 41 Pro-Clinton-(Anti-Trump)-Fake-News mit 7,6 Millionen Shares insgesamt also schlechter dargestellt worden, zumal viele der Pro-Trump-Fake-News Geschichten in Wirklichkeit Hillary Clinton diffamierten \autocite[vgl.][212 \psqq]{allcott2017social}. Unabhängig davon, ob negative Fake News sich letztendlich als falsch herausstellen, rücken sie das Subjekt der Geschichte in ein negatives Licht, vergiften die sachliche politische Diskussion und es bleibt fraglich, ob nach dem Widerlegen der Theorie der Gegenbeweis genauso weitreichend geteilt wird wie die ursprüngliche Geschichte. Im schlimmsten Fall bleiben zweifelnde Wähler weiterhin skeptisch oder lassen sich durch Desinformationen von der Wahl abhalten. 

Nicht nur waren Fake News 2016 auf Social Media weit verbreitet, sondern der Zugriff auf Fake-News-Websiten ist außerdem überproportional über Social-Media-Verlinkungen ausgelöst worden. Während 10.1\% der Besucher von seriösen, etablierten Nachrichtenseiten auf diese über Social-Media-Verlinkungen gelangt sind, waren es bei Fake-News-Seiten 41,8\% der Besucher, die über Social Media auf diese gelangten. Somit war nicht nur die Verbreitung von Fake News in der US-Wahl 2016 nennenswert, sondern zusätzlich auch die zentrale Rolle von Social Media als Vermittler zu diesen Fake-News-Seiten \autocite[vgl.][222]{allcott2017social} (\autocite[]{fig:newsSites}).

Zuletzt sei die Einmischung anderer Nationalstaaten zu erwähnen, so mündeten die Ereignisse um die US-Wahl 2016 in eine Ermittlung des amerikanischen „U.S. Senate Select Committee on Intelligence“ zu potenzieller russischer Einmischung in den US-Wahlkampf. Das Ergebnis lautete: “The Committee found that the Russian government engaged in an aggressive, multifaceted effort to influence, or attempt to influence, the outcome of the 2016 presidential election.” \autocite[V]{SenateRussian}. Die Einflussnahme fand überwiegend digital und letztendlich auch auf Social Media statt \autocite[vgl.][185, 190]{SenateRussian}. Vor allem der Einsatz von russischen Bots ist wohldokumentiert. So kam eine Analyse der Twitter-Accounts, die laut der US-Ermittlungen mit russischen Netzwerken in Verbindung standen, zu dem Schluss, dass die russischen Bots überwiegend konservative Pro-Trump-Inhalte verbreiteten. Retweeted wurden sie vor allem von Pro-Trump-Nutzern, die somit den Bots eine Stimme gaben \autocite[vgl.][]{badawy2018analyzing}. Twitter-Bots sind in der Regel effektiv, um Menschen zu beeinflussen, die der gleichen ideologischen Ansicht sind wie der Bot, was im Falle einiger Republikaner zu einer weiteren Polarisierung geführt hat.

Aus der obigen Betrachtung folgt, dass es verschiedene Aspekte der politischen Öffentlichkeit auf Social Media im Rahmen der US-Wahl 2016 gab, die einen Einfluss auf die politische Meinungsbildung zugunsten Donald Trumps hatten. Ob Fake News, Bots oder die Arbeit von CA sowie weitere Faktoren, wie Trumps „twitter-first“-Ankündigungen \autocite[vgl.][22]{groshek2017helping} zum Volk, letztendlich wahlentscheidend waren, lässt sich allerdings nicht feststellen. So gibt es zwar Stimmen, die behaupten, ohne Social Media hätte Donald Trump die Wahl nicht gewonnen \autocite[vgl.][]{GuardianParkinsonFN}\autocite[vgl.][]{WashingtonDeweyFN}\autocite[vgl.][]{NYMagReadFN}, jedoch ist Social Media nicht das einzige Medium, über das politische Informationen ausgetauscht werden und auch nicht der einzige Weg, um Wählerstimmen zu gewinnen. 


\subsection{Übertragung auf die Bundestagswahl 2021}
Für eine Zukunftsprognose in Deutschland bietet sich die Bundestagswahl 2021 als nächste große deutsche Wahl an. Es wird hierbei erörtert, inwiefern die Beeinflussung, die zuerst im Allgemeinen (s. Kapitel 3.2) und dann anhand des Beispieles der US-Wahl 2016 (s. Kapitel 3.3) erläutert wurde, auch auf Deutschland zutreffen könnte.

Für einen zunehmenden Einfluss von Social Media bei deutschen Wahlen spricht, dass der politische Meinungsaustausch auf diesen Plattformen zunimmt (s. Kapitel 3.1). Je mehr politische Inhalte dort ausgetauscht werden und sich das Geschehen dorthin verlagert, desto leichter ist dort eine Beeinflussung möglich. 
Die hierfür nötigen Methodiken, die bei der US-Wahl 2016 bereits zum Einsatz kamen (s. Kapitel 3.2), sind auch in Deutschland möglich. So kann sich der Staat nicht dagegen wehren, dass Bots oder Fake News Einfluss auf die Meinungen der Wähler nehmen. Zwar sind Bots laut Nutzungsbedingungen der Plattformen \autocite[vgl.][]{TwitterRules}\autocite[vgl.][]{NutzungsbFacebook}\autocite[vgl.][]{NutzungsbYouTube} verboten, da jeder Nutzer nur einen Account haben darf, allerdings ist hierbei die dauerhafte Überprüfung aller Accounts aufgrund der enormen Anzahl (bei Facebook 2,7 Milliarden) \autocite[vgl.][]{SMNutzer} nur schwer möglich. Selbiges gilt für Fake News, die nach den Nutzungsbedingungen verboten sind, falls sie Verleumdungen beinhalten (s. Kapitel 4.2). So sollen diese seit 2017 nach dem Netzwerkdurchsetzungsgesetz gelöscht werden.
Auch können die Nutzer leicht in die bereits erwähnten Netzwerke (s. Kapitel 3.2), die mit Filterblasen gleichzustellen sind, gelangen, sodass sie fast ausschließlich politische Inhalte einer Partei oder politischen Ausrichtung erhalten. Dies kann ihr Wahlverhalten maßgeblich beeinflussen.
Schlussendlich hat sich auch bereits bei der Bundestagswahl 2017 gezeigt, dass Microtargeting auch in Deutschland möglich ist. So wurden potenzielle AfD-Wähler gezielt mit Wahlwerbung von Jens Spahn konfrontiert, um sie von einer Wahl der AfD abzuhalten und sie für die CDU zu gewinnen \autocite[vgl.][5]{microtargeting}.

Allerdings wurden seitdem die Voraussetzungen für politische Werbung auf Facebook verschärft. So muss zum Beispiel gekennzeichnet sein, wer die Werbung finanziert und um Werbeanzeigen schalten zu können, muss man vorher ein Autorisierungsverfahren durchlaufen \autocite[vgl.][]{facebookPolitischeWerbung}\autocite[vgl.][]{zeitFacebookWahlwerbung}. 
Auf Twitter ist Wahlwerbung derzeit vollständig verboten \autocite[vgl.][]{twitterPolitischerInhalt}. Hier darf nicht für Kandidaten oder Parteien geworben werden. Auf diesen Plattformen kann eine Beeinflussung durch Werbung in dem oben genannten Maß nicht mehr stattfinden.
Im Allgemeinen ist Microtargeting, also das gezielte Schalten von Werbung, in Deutschland aufgrund der DSGVO nur unter strengen Voraussetzungen und somit nicht wie bei der US-Wahl 2016 möglich \autocite[vgl.][1,11]{microtargeting}. Dies ist zusätzlich dadurch erschwert, dass kaum Daten über Wählergruppen gesammelt wurden \autocite[vgl.][4,5]{microtargeting}.

Ferner liegt im amerikanischen Zweiparteiensystem eine Art \glqq Grundpolarisierung\grqq{} in Republikaner und Demokraten vor, wohingegen im deutschen Parteienpluralismus eine größere Parteienvielfalt herrscht. Während negative Nachrichten bezüglich des Gegners in den USA direkt einen Vorteil für die eigene Partei bedeuten, ist dieser direkte Bezug aufgrund der fehlenden Polarisierung in nur zwei Gruppen in Deutschland nicht vorhanden. Somit ist die Beeinflussung im Bezug auf die gesamte Parteienlandschaft in Deutschland schwieriger und fällt dementsprechend bei gleichem Einsatz schwächer aus als in den USA.

Aus unserer Umfrage geht außerdem hervor, dass Social Media als Quelle für politische Inhalte unter den Teilnehmern als nicht sehr seriös eingeschätzt wird. Auf einer Skala von 0 (nicht seriös) bis 5 (sehr seriös) ergab sich ein Durchschnittswert von 1,95, der somit weit unter dem Mittelwert von 2,5 liegt (\autocite[vgl.][]{fig:socialMediaSeriös}). Auch wenn man einer Beeinflussung durch Inhalte auf Social Media nie vollständig entgehen kann, wissen die Teilnehmer im Schnitt also, dass mit den Inhalten kritisch umzugehen ist.
Schlussendlich ist Social Media nicht so weit verbreitet wie in den USA, was die Nutzung für Nachrichten anbelangt. Während in den USA 18\% der Befragten ihre Nachrichten hauptsächlich durch Social Media bezogen \autocite[vgl.][]{AmericansSourcesNews}, sind es bei der ähnlichen Frage in Deutschland nur 1\% \autocite[vgl.][]{GerSourcesNews}. Auch wenn die Umfrage in Amerika ein Jahr vor der in Deutschland durchgeführt wurde, ist in Amerika seitdem eher eine Steigerung des Prozentsatzes als eine Angleichung an das Ergebnis in Deutschland zu erwarten (s. Kapitel 3.1).

Zusammenfassend ist eine Beeinflussung der Bundestagswahl 2021 über Social Media nicht auszuschließen, welche aber wahrscheinlich geringer ausfallen wie in der US-Wahl 2016 wird.


\section{Verantwortung der IT-Konzerne}

\subsection{Rechtliche Verantwortung}
Rechtlich gesehen sind die genannten Methodiken (s. Kapitel 3.2) in Deutschland nicht grundsätzlich verboten. Nach der Datenschutzgrundverordnung ist das Erheben und Verarbeiten von personenbezogenen Daten zulässig, falls dies automatisch und mit Einwilligung, die man durch Akzeptieren der Nutzungsbedingungen gibt, geschieht (Art. 20 \S 1 DSGVO). Jedoch dürfen hierdurch nicht Rechte oder Freiheiten der Nutzer eingeschränkt werden (Art. 20 \S 4 DSGVO).  (Art.3 GG). Generell ist somit nicht verboten, Daten auszuwerten und zum Beispiel Bonitäten als Zahlen zu errechnen, da hier im Folgenden noch eine Bewertung durch einen Menschen erfolgen muss\autocite[vgl.][7]{martini2017algorithmen}. Allerdings ist das damit verbundene Profiling, also das Bewerten des Nutzers, verboten, falls dies negative Konsequenzen für den Betroffenene hat. Wird ein Bewerber ausschließlich aufgrund von Profiling nicht eingestellt, ist dies rechtswidrig \autocite[vgl.][9]{martini2014big}.

Was die Inhalte anbelangt, ist jeder mündige Bürger stets selbst für sein Handeln verantwortlich. Dies gilt auch im Internet und insbesondere auf Social-Media-Plattformen. So kann zum Beispiel der Verfasser von Tweets, Kommentaren oder Retweets zur Verantwortung gezogen werden, wenn diese Beleidigungen, üble Nachrede oder Verleumdung beinhalten (\S 185, 186, 187 StGB).

Allerdings ist es für Opfer online schwieriger, Straftaten zu melden, sodass die Täter rechtlich belangt werden können, da der Täter durch einen frei gewählten Benutzernamen meist anonym bleiben kann.

Um den Opferschutz auch online gewähren zu können, trat am 01.10.2017 das Netzwerkdurchsetzungsgesetz (im Folgenden als \glqq NDG\grqq{} bezeichnet) in Kraft \autocite[vgl.][Art. 3]{NetzDGOfficial}. Dies legt fest, zu welchen Handlungen die Plattformbetreiber verpflichtet sind. So muss jede Plattform eine Beschwerde- und Meldestelle für die Nutzer einrichten, bei der Kommentare, Videos oder Ähnliches gemeldet werden und von den Plattformbetreibern überprüft werden \autocite[vgl.][Art. 1 \S 3]{NetzDGOfficial}. Gleichzeitig müssen die Betreiber dieser sozialen Netzwerke halbjährliche Berichte verfassen, die ihre Maßnahmen gegen Straftaten beschreiben \autocite[vgl.][Art. 1 \S 2]{NetzDGOfficial}. Des Weiteren sollen offensichtlich rechtswidrige Inhalte, die also beispielsweise die oben genannten Straftaten umfassen (\S 185, 186, 187 StGB) innerhalb von 24 Stunden gelöscht werden \autocite[vgl.][Art. 1 \S 3 Abs. 2]{NetzDGOfficial}. Muss der Sachverhalt erst geprüft werden, muss die Löschung innerhalb von 7 Tagen erfolgen, wenn die Inhalte sich als rechtswidrig herausstellen \autocite[vgl.][Art. 1 \S 3 Abs. 2]{NetzDGOfficial}. Nichtbeachtung des NDGs führt zu Bußgeldverfahren \autocite[vgl.][Art. 1 \S 4]{NetzDGOfficial}. Während das NDG zum einen den Opfern Handlungsmöglichkeiten bietet und Plattformen verpflichtet, rechtswidrige Inhalte zu löschen, ergeben sich zum anderen allerdings neue Probleme. 

Da die Löschung für die Plattformen verpflichtend ist, müssen die Betreiber bei jeder Meldung in kurzer Zeit die Inhalte auf Straftaten überprüfen. Es müssen somit juristische Entscheidungen von wirtschaftlich denkenden Unternehmen getroffen werden. Um den Anforderungen gerecht zu werden, muss neues Personal zur Prüfung eingestellt werden. Gleichzeitig bedeutet die Löschung jedes Inhaltes, dass die Plattform weniger Views erhält und somit weniger verdient. Das NDG ist also mit Kosten für das Unternehmen verbunden, die die Betreiber so niedrig wie möglich halten wollen. Dies könnte wiederum dazu führen, dass nicht alle rechtswidrigen Kommentare gelöscht werden. 

Für eine Löschung von Inhalten oder Sperrung von Accounts bedarf es nicht zwingend einer Meldung durch Benutzer. Alle Nutzer verpflichten sich beim Eröffnen eines Accounts gegenüber den Nutzungsbedingungen  der jeweiligen Plattform (Facebook und Instagram \autocite[vgl.][]{NutzungsbFacebook} , Twitter \autocite[vgl.][]{TwitterAlg}, YouTube \autocite[vgl.][]{NutzungsbYouTube}). So sind zum Beispiel die Androhung von Gewalt \autocite[vgl.][]{NutzungsbFacebook}, \autocite{TwitterRules} oder Inhalte, die Dritten schaden \autocite[vgl.][]{NutzungsbYouTube}, verboten. Verstoßen die Nutzer gegen die Regeln, dürfen die Plattformbetreiber die Inhalte löschen oder die Accounts vorübergehend oder dauerhaft sperren \autocite[vgl.][]{NutzungsbFacebook}\autocite[vgl.][]{NutzungsbYouTube}\autocite[vgl.][]{TwitterDurchsetzung}. Eine Sperrung oder Löschung kann auch im Eigeninteresse der Plattformen liegen, falls diese unter dem Druck der Öffentlichkeit stehen. So kann eine zu große Ignoranz gegenüber Inhalten, die an der Grenze des Rechtlichen liegen, den Ruf der Plattform schädigen.

Gleichzeitig ist aber jede Sperrung oder Löschung eine Einschränkung der Freiheit der Nutzer. Ein Fall einer Sperrung eines Twitter-Accounts endete beispielsweise vor dem Landgericht in Nürnberg. Der Konzern hatte den Account nach dem Tweet „Aktueller Anlass: Dringende Wahlempfehlung für alle AfD-Wähler. Unbedingt den Stimmzettel unterschreiben. ;-)“ gesperrt \autocite[vgl.][]{GerichtsbeschlussSperrungTwitter}. Ob der Tweet angreifend oder satirisch gemeint war, ist unbekannt, wobei der Grat zwischen dem einen und dem anderen, zwischen einer Straftat und rechtmäßigem Inhalt, schmal ist. Die Sperrung wurde schlussendlich nach einer Entscheidung des Landgerichts aufgehoben, da sie gegen die Meinungsfreiheit (Art. 5 Abs. 1 GG) verstoße.

Die Konzerne tragen noch immer selbst die Verantwortung, welche Inhalte nach dem NDG gelöscht werden müssen, sodass es zu Konflikten und Diskussionen bezüglich Sperrungen oder Löschungen kommt (s. obige Beispiele). Dies liegt, wie bei der Sperrung Trumps, (s. Kapitel 4.2) nicht zwingend im Interesse der deutschen Regierung. Um die Möglichkeiten der Einflussnahme auf die politische Meinungsbildung einzugrenzen, müssen die geltenden Regelungen und Gesetze also erweitert werden. Da regional unterschiedliche Gesetze hierbei für Konzerne Mehraufwand und weitere Probleme durch keine weltweit eindeutige Regelung bedeuten können, arbeitet die EU derzeit an einem europaweit einheitlichen Gesetz für rechtliche Grundlagen auf Social Media, dem \glqq Digital Services Act\grqq{} \autocite[vgl.][]{DigitalServicesAct}. Genaue Inhalte sind aber noch nicht ausformuliert. Ob der Digital Services Act lücken- und problemloser wirkt als z.B. das NDG in Deutschland, wird sich zeigen. 

Da der rechtliche Rahmen derzeit nur begrenzt und mit einigen Lücken definiert ist, kann die Verantwortung der IT-Konzerne durch die rechtlichen Aspekte allein nicht genau festgelegt werden. Hier kann eine Debatte der ethischen Vertretbarkeit Abhilfe leisten, die Inhalt des nächsten Kapitels ist.


\subsection{Soziale und ethische Verantwortung}

Generell ist zu sagen, dass die Risiken des Missbrauchs auf Social-Media-Plattformen mit den Vorteilen, die die Plattformen bringen, im Verhältnis zueinander betrachtet werden müssen. So besteht zwar das Risiko, dass die sachliche politische Atmosphäre, wie in der US-Wahl 2016, eingeschränkt wird (s. Kapitel 3.3), jedoch bringt Social Media auch viel Potential für die Bereicherung des politischen Austausches mit. Zum einen ermöglicht es einer Vielzahl von Nutzern die freie Meinungsäußerung und den Austausch von Meinungen untereinander und kann somit die politische Partizipation unter anderem auch in Ländern mit eingeschränkter Meinungsfreiheit erleichtern. Dies ist im Sinne des Artikels 10 der GI-Leitlinien, die festlegen, dass \glqq IT-Systeme zur Verbesserung der lokalen und globalen Lebensbedingungen beitragen [sollen]\grqq{} \autocite[vgl.][Art. 10]{LeitlinienGI}. 
Da somit eine wichtige Absicht der Plattformen ist, die Meinungsfreiheit zu ermöglichen, ist das Handeln der Social-Media-Konzerne deontologisch gesehen durchaus vertretbar. 

Aus der ethischen Sicht des Utilitarismus ist das Betreiben von Social-Media-Plattformen genau dann vertretbar, wenn die positiven Konsequenzen, wie eine bessere politische Partizipation und Meinungsbildung für möglichst viele Nutzer, gegenüber den negativen Konsequenzen wie Manipulation überwiegen. Damit das \glqq größtmögliche Wohl\grqq{}, also ein guter, sachlicher und differenzierter Zugang zur Politik, für die größtmögliche Anzahl an Nutzern erfüllt wird, müssen Maßnahmen gegen Missbrauch (zum Beispiel Fake-News und Bots) erfolgen.

Die Fragestellung, die hierbei aufkommt, ist, wer die Maßnahmen entwickeln und durchsetzen muss. Für das Löschen von rechtswidrigen Inhalten sieht laut Netzwerkdurchsetzungsgesetz der Gesetzgeber die Unternehmen  in der Pflicht (s. Kapitel 4.1). Jedoch gibt es auf Social Media auch Fake-News, die nicht strafrechtlich zu belangen sind.

Hierbei handelt es sich um keine rechtliche Verantwortung, sondern um eine ethische, da Falschmeldungen gegen Werte wie Ehrlichkeit verstoßen. Aus dem 10. Artikel der GI-Leitlinien folgt, dass der Betreiber des IT-Systems \glqq die soziale und gesellschaftliche Verantwortung für die Auswirkungen\grqq{} seines IT-Systems trägt \autocite[vgl.][Art. 10]{LeitlinienGI}. Somit sind es zwar nicht die Betreiber selbst, die illegale oder falsche Inhalte auf ihren Plattformen posten, jedoch tragen sie die Verantwortung dafür, dass den Inhalten eine Bühne gegeben wird. Daraus folgt, dass die Unternehmen hinter Social Media auch mitverantwortlich sind, wenn durch die Inhalte auf ihren Plattformen eine politische Beeinflussung beziehungsweise Manipulation stattfindet. 

Für die Unternehmen spricht, dass es diverse Ansätze gibt, die den politischen Missbrauch der Plattformen einschränken sollen. Nach der fragwürdigen Nutzung von Microtargeting \autocite[vgl.][]{MicrotargetingBPB} in Kombination mit diffamierenden Inhalten in der US-Wahl 2016 hat Twitter, wie bereits erwähnt, inzwischen Wahlwerbung deaktiviert \autocite[vgl.][]{twitterPolitischerInhalt} und auch auf Facebook wurden die Richtlinien verschärft \autocite{facebookPolitischeWerbung}(s. Kapitel 3.4). Ferner betreibt Facebook seit 2016 ein \glqq Faktenprüfungsprogramm\grqq{} in Kooperation mit dem \glqq International Fact-Checking Network\grqq{} \autocite[vgl.][]{FactCheckFB}, welches möglicherweise falsche Inhalte als \glqq umstritten\grqq{} kennzeichnet \autocite[vgl.][]{FactCheckFBAlert}. Somit gibt es durchaus Bemühungen, der eigenen Verantwortung nachzukommen.

Beim Nachkommen der Verantwortung gibt es allerdings Sachverhalte, die aus ethischer Sicht kritisch zu betrachten sind. Hierzu zählt beispielsweise die Sperrung von Donald Trumps Social-Media-Accounts. 
Auf den drei Plattformen Facebook, Instagram und Twitter kann er derzeit keine Posts absetzen \autocite[vgl.][]{BanTrumpFBInsta}\autocite[vgl.][]{BanTrumpTwitter}. Begründet wird die Sperrung jeweils mit einem zu hohen Risiko, was von ihm ausgehe. So sei besonders nach den Angriffen auf das Kapitol eine Sperrung unvermeidlich gewesen. Der deutsche Regierungssprecher Seibert warnte jedoch vor der Sperrung der Accounts von Trump. So sollten Eingriffe in die Meinungsfreiheit wie dieser nur im Rahmen der Gesetze vorgenommen und nicht durch die Unternehmen vollzogen werden. Nichtsdestotrotz sei die Verantwortung der Unternehmen derzeit groß, dass die \glqq politische Kommunikation nicht vergiftet wird durch Hass, durch Lüge, durch Anstiftung zur Gewalt\grqq{} \autocite[vgl.][]{ARDMerkelTrump} \autocite[vgl.][]{ZDFMerkelTrump}. Die Sperrung Trumps ist derzeit also eine von vielen rechtlichen und ethischen Grauzonen bei Social-Media-Plattformen. 

Es wurde also bereits festgestellt, dass die Existenz der Social-Media-Plattformen die politische Meinungsbildung beeinflussen kann und dass die Konzerne für die Auswirkung in Teilen verantwortlich sind. Die Beeinflussung umfasst aber zusätzlich auch die Methodiken (s. Kapitel 3.2), die von den Plattformen eingesetzt werden.

Die wichtigste Beeinflussungsmöglichkeit sind hierbei die Algorithmen, die den Nutzern nach bestimmten Verfahren Inhalte vorschlagen, die ihre politische Meinung beeinflussen können. Zur Relativierung der Verantwortung muss hierbei zuerst gesagt werden, dass jedem Nutzer die Inhalte angezeigt werden, die ihn höchstwahrscheinlich auf Grundlage des erstellten Eigenschaftenprofils interessieren (s. Kapitel 3.2). Die Zuordnung der Eigenschaften passiert hierbei automatisch und beruht auf bereits erfassten Daten über den Nutzer, die auch seine Interessen ausdrücken. Werden dem Nutzer durch den Algorithmus politische Inhalte einer bestimmten Richtung oder Partei vorgeschlagen, ist dies durch seine vorherige Nutzung begründet. Der Nutzer erhält also auch genau die politischen Inhalte, die ihn interessieren.

Außerdem verarbeiten die Algorithmen, wie erwähnt, nur vorhandene Daten auf automatischem Weg und sind nicht von persönlichen politischen Interessen der Plattformbetreiber beeinflusst. So empfehlen die Algorithmen zwar eher kontrovers diskutierte politische Inhalte, es ist dabei aber nicht von Bedeutung, zu welcher politischen Orientierung diese Inhalte passen \autocite[vgl.][198,199]{wylie2019mindf}.

Allerdings findet trotzdem eine Beeinflussung durch das Unterstützen kontroverser Inhalte \autocite[vgl.][198,199]{wylie2019mindf}, wozu auch populistische zählen, statt. 

Ein Indiz dafür sind die Interaktionen zu Beiträgen von politischen Parteien \autocite[vgl.][]{InteraktionenSM} auf Social Media in Relation zum Wahlergebnis \autocite[vgl.][]{Wahl2017Ergebnis} gesetzt. So erhält man hierbei den höchsten Grad der Interaktionen bei den Parteien AfD, FDP und Die Linke, die an den Rändern des von der Friedrich-Ebert-Stiftung erstellten Parteienspektrums einzuordnen sind \autocite[vgl.][]{ParteienspektrumD} (\autocite[]{tab:InteraktionenWahl2017}). Somit fördern die IT-Konzerne durch die Beeinflussung mithilfe von Algorithmen extremere politische Ausrichtungen, was zu einer Spaltung der politischen Landschaft führen kann. 
\section{Fazit}
Nicht zuletzt ermöglichen die Algorithmen eine Einordnung in sogenannte Netzwerke (s. Kapitel 3.2). Dies führt im politischen Zusammenhang dazu, dass der Nutzer gezielt mit ähnlichen politischen Inhalten konfrontiert wird und somit in seiner politischen Ausrichtung, die möglicherweise erst durch die Netzwerke hervorgerufen wurde, bestärkt wird. Eine weitere Folge ist, dass Nutzer Probleme damit bekommen können, sich differenziert mit anderen politischen Meinungen auseinanderzusetzen, wenn ihnen diese nicht mehr angezeigt werden. 

Utilitaristisch, also die Folgen der Beeinflussungsmöglichkeiten betreffend, ist die ethische Verantwortung der Konzerne, wie gerade gezeigt, hoch. So können die Algorithmen darüber entscheiden, mit wessen Inhalten der Nutzer ausschließlich konfrontiert wird und dadurch das gesamte politische Bild verändern. 



Zusammenfassend wurde festgestellt, dass den Unternehmen, die Social-Media-Plattformen betreiben, rechtlich und ethisch gesehen eine große Verantwortung auferlegt wird.
Hierfür wurde zuerst die politische Beeinflussung durch Social Media aufgezeigt. Grundlage dafür ist, dass Social Media im Allgemeinen, aber auch insbesondere im politischem Bezug, stetig an Relevanz dazugewinnt. So sind seit ungefähr zehn Jahren alle großen deutschen Parteien auf den betrachteten Social-Media-Plattformen vertreten. Des Weiteren ist die Nutzung von Social Media für politische Nachrichten vor allem unter den jüngeren Nutzern hoch. Da sich dieser Trend auch in Zukunft fortsetzen wird, gewinnt Social Media weiterhin an Relevanz und besitzt somit auch immer mehr Einfluss auf die politische Meinungsbildung (s. Kapitel 3.1). 
Es wurde außerdem dargestellt, mit welchen Methodiken eben dieser Einfluss ausgeübt wird. So erfolgte die Beeinflussung beispielsweise durch Microtargeting, Algorithmen, Netzwerke, Fake News und Bots und auch eine Einmischung anderer Staaten ist möglich (s. Kapitel 3.2). Letzteres wurde bereits für die US-Präsidentschaftswahl 2016 offiziell bestätigt. Aber auch die anderen Methodiken fanden dort Verwendung. Ob die Beeinflussung auf Social Media letztendlich wahlentscheidend war, kann allerdings nicht festgestellt werden (s. Kapitel 3.3).
Eine Anwendung der gewonnen Erkenntnisse auf die anstehende Bundestagswahl 2021 ergibt, dass der Einfluss von Social Media dort wahrscheinlich geringer sein wird als bei der US-Wahl 2016. Hierfür spricht unter anderem, dass die Methodiken in Deutschland strenger reguliert sind und Social Media im Allgemeinen zu politischen Zwecken weniger weit verbreitet ist (s. Kapitel 3.4).

Nachdem durch die Analyse bestätigt wurde, dass Social Media Einfluss auf die politische Meinungsbildung hat, wurde erörtert, inwiefern die Konzerne, die die Plattformen betreiben, für diese Beeinflussung verantwortlich sind.

Rechtlich gesehen ist die Verantwortung zuerst dadurch beschränkt, dass jeder Nutzer stets selbst für die Inhalte verantwortlich ist. Allerdings verpflichtet das Netzwerkdurchsetzungsgesetz die Plattformbetreiber dazu, rechtswidrige Inhalte zu löschen. Was rechtswidrig ist und was nicht, muss allerdings von den Plattformbetreibern (also nicht von staatlichen Organen der Judikative) entschieden werden. Somit befinden sich die damit verbundenen Sperrungen und Löschungen oftmals in rechtlichen Grauzonen, wodurch die Eigenverantwortung der Konzerne sehr groß ist (s. Kapitel 4.1).

Ethisch betrachtet ist festzuhalten, dass die Social-Media-Konzerne nach der deontologischen Ethik durch die Bereitstellung ihrer Plattform mit dem richtigen Motiv handeln, da sie die freie Meinungsäußerung ermöglichen oder erleichtern. Nach dem utilitaristischen Ansatz ist die Beeinflussung jedoch problematisch zu betrachten. Danach ist die Plattform dafür verantwortlich, dass die politische Meinungsbildung durch Priorisieren von Inhalten verzerrt wird. Die hierbei betrachteten Maßnahmen können unter anderem zu einer stark eingeschränkten politischen Sichtweise führen (s. Kapitel 4.2). 

Sowohl rechtlich als auch ethisch ist die Eigenverantwortung der Social-Media-Konzerne unter anderem aufgrund der zum Teil großflächigen rechtlichen Grauzonen groß und wird in den nächsten Jahren bei ähnlichen Voraussetzungen noch gesteigert werden. Dies kann unter Umständen zu einem Machtmissbrauch führen, der eine politische Beeinflussung herbeiführt.

Um diesem Risiko entgegenzuwirken, kann der Gesetzgeber die Verantwortung von den Konzernen hin zum Staat verlagern. So können Gesetze in Anlehnung an das NDG präziser festlegen, welche Inhalte gelöscht und welche Nutzer gesperrt werden sollen. Die Regulierung von Wahlwerbung, welche die Plattformen in Teilen selbstständig formulieren, kann genauso wie die Pflicht zu Faktenchecks und Kennzeichnung von Fake News gesetzlich niedergeschrieben werden. Um die Unternehmen von der Verantwortung zu befreien, juristische Entscheidungen zu treffen, kann die Prüfung der Inhalte beispielsweise in Kooperation mit staatlichen Instanzen erfolgen. Schlussendlich gilt wie oftmals: Die Plattform wird durch das Unwissen der Nutzer für den Prozess der politischen Meinungsbildung gefährlich. Hier kann die frühe Aufklärung in der Schule Abhilfe leisten. 

Damit Social Media in Zukunft also als legitime Quelle für politische Informationen ohne ein hohes Risiko der Beeinflussung genutzt werden kann, muss der Staat zusätzliche Regulierungen einführen und die Unternehmen müssen angesichts ihrer Verantwortung aus Eigeninitiative stärker gegen den Missbrauch ihrer Plattformen vorgehen.

